\section{Introduction}

Artificial intelligents could be roughly divided into two classes: those that are trying to understand the world (classification, clustering, etc.) and those that are trying to improve the world with actions (decision making, planning, etc.). Reinforcement learning aims to do both, since it's trying to make optimal decisions in an unknown world. By doing some episodes of trials in the world, an RL agent collects trajectories of states (i.e. observations), rewards and actions. With these trajectories, the RL agent can pick the best action in any observed state, which would gain him the most reward. RL differs from decision making methods like Marcov decision processes, because MDPs have a perfect understanding and observation of the world, while RL agents know nothing about the transition model of states or the reward model (i.e., the rewards related to each state). In real life situations, the actual states are often not observable, and the agents would also need to learn to infer the states with observations.

Education, or to be more specific, intelligent tutor, is a perfect case where RL methods can be applied. We can easily draw some parallels between tutor systems and RL agents. The actions are the various tutoring materials that could be presented to students, while the rewards could be students' performance in tests. The world is also unknown: the transition model of students' knowledge states (i.e. how students would digest the materials) is unknown, and the reward model is also unknown (i.e. how student with a certain knowledge state would perform in a test). In education, we are also dealing with a partially observable environments, since we won't be able to directly see the knowledge state for students, and we can only have a belief state infered from observations of the students.

Education also poses an even larger challenge than other RL tasks. In conventional RL tasks like robotics, the robot can run trials in the real environments as much as we want, since in most cases the worst we can expect is merely a broken mechanical arm resulted from tripping over a stone. In education, it won't be ethical to collect online data on real students with an agent that's still learning from trials and errors. A crappy work-in-progress would do irreversible harms to real kids in this practice, and the things that could be broken would be the kids' future. Therefore, doing offline policy evaluation with trajectories collected from real world tutor systems (in which runs policies either designed by expert humans or well-tested AI agents) is an important track of research in educational RL.

Prof. Emma Brunskill's group, in which I interned for the summer, lies its interest in RL and education. We have shown that these two areas are intrinstically related. Our group covers a wide spectrum of research areas, from RL theories (machine learning) and traditional AI, all the way to more application oriented tracks like data mining and human computer interaction (in education). Given the special, yet not unique challenge of education described previously, our research projects are mostly focused on offline policy evaluation or planning. Educational data mining, which focuses on data collected from intelligent tutoring systems, is one of the may side interests of our group. Several distinctive characteristics of EDM include dealing with sequential data, and a special focus on discovering the latent hierarchical structure of data items (i.e. prerequisite structure of knowledge components). Deep learning, while has been widely adopted in various areas like computer vision, is still an emerging methods in RL and EDM, where only only a handful of previous works could be found. It is also a new area for the group.

In this section, we have briefly described the backgrounds and motivations of the projects. The following sections are organised as: section \ref{sec:DKT} described a deep learning model which my work in all 3 projects are based on; section \ref{sec:deepRL}, section \ref{sec:fraction} and section \ref{sec:influence} discusses the big pictures of the three projects, where we are and what part I play in them. At last, section \ref{sec:life} jots down some other aspects of the internship basides work, and section \ref{sec:summary} summarizes the 10-week internship.









